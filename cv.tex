\documentclass[a4paper,11pt,sans]{moderncv}
\usepackage[english]{babel}
\usepackage[T1]{fontenc}
\usepackage[utf8]{inputenc}
\usepackage{lmodern}
\usepackage{microtype}

\firstname{Mattijs}
\familyname{Korpershoek}
\title{\large{Embedded Android/Linux software engineer}}
\address{668 Willowgate St}{Mountain View}{94043, CA}
\phone[mobile]{+1~(415)~926~2483}
\email{mattijs.korpershoek@gmail.com}
\social[linkedin]{mattijskorpershoek}
\social[github]{Makohoek}
%\photo{example.jpg}

\moderncvstyle{fancy}
\moderncvcolor{blue}                               % color options 'blue' (default), 'orange', 'green', 'red', 'purple', 'grey' and 'black'
\nopagenumbers{}
\setlength{\hintscolumnwidth}{35mm}
%\setlength{\hintscolumnwidth}{15mm}
% adjust the page margins
\usepackage[scale=0.85]{geometry}
\AtBeginDocument{\recomputelengths}
\moderncvicons{awesome}


\begin{document}

\maketitle

\section{Summary}
I am a software developer and passionated by technology since I was ten years old.
I do platform development since 2014 on Android related products, with focus on smart-watches.
I shine in learning and sharing technical knowledge.
I was born in the Netherlands then got my Master Degree in France.
Therefore I speak fluent English, French and Dutch.

\section{Technical skills}
\begin{cvcolumns}
  \cvcolumn{Languages}{%
    \begin{itemize}
    \item C
    \item C++
    \item Python
    \item Java
    \item Bash
    \end{itemize}
  }
  \cvcolumn{Android}{%
    \begin{itemize}
      \item Platform development
      \item Hardware bring-up
      \item New product enabling
      \item System debug
      \item Optimization
    \end{itemize}
  }
  \cvcolumn{Other}{%
    \begin{itemize}
      \item Linux kernel \& userspace
      \item Git
      \item Gerrit
      \item Spacemacs
    \end{itemize}
  }
\end{cvcolumns}
    
\section{Key accomplishments}
\begin{cvcolumns}
  \cvcolumn{Audio software stack}{%
    On Android Wear based product, I was the audio software stack owner.
    I have developed, debugged, and fixed in all the software stack. This includes DSP firmware, Linux kernel drivers, Audio HALs and apps (java) level.
  }
\end{cvcolumns}
\begin{cvcolumns}
  \cvcolumn{Manufacturing OS design/implementation}{%
    Traveled to China for pre-mass production, on-site troubleshooting and support.
    Developed Intel wearable Manufacturing OS, for hardware testing on the production line.
  }
  \cvcolumn{Platform bring-up}{%
    Traveled to the Bay Area for a week for a hard-ware bring-up. Audio stack (with codec, speakers, four microphones) was up in two days and had three more days to make the camera work.
  }
\end{cvcolumns}


\section{Work history}
\cventry{February, 2017 -- present}{Linux/Android R\&D for Intel New Device Group}{CELAD}
{San Francisco Bay Area}{USA}
{
    \begin{itemize}
        \item Android/Linux drivers, Android Framework / HALs
        \item Production line support, system debugging, hardware bring-ups
        \item C, C++, Java, Git, Python, XML, Bash, Android build system
    \end{itemize}
}

\cventry{September, 2014 -- January, 2017}{Linux/Android R\&D for Intel New Device Group}{CELAD}
{Toulouse}{France}
{
    \begin{itemize}
        \item Android/Linux drivers, Android Framework / HALs
        \item Production line support, system debugging, hardware bring-ups
        \item Android/Linux drivers (ASoC framework)
        \item Android audio HAL, AudioFlinger, AudioPolicy and DSP Firmware
        \item C, C++, Git, Python, XML, Bash, Android build system
    \end{itemize}
}

\cventry{April, 2014 -- August, 2014}{Internship: Opensource audio HAL component}{CELAD}
{Toulouse}{France}
{
  Worked on open-sourcing the Parameter-Framework, a major component of Intel's Android Audio HAL.
  This component is now part of Android AOSP (/external/parameter-framework/)
}

\cventry{May, 2013 -- August, 2013}{Internship: embedded and web development}{24green}
{Vlaardingen}{The Netherlands}
{
  Developed a web (REST) API which handles climate control functions in green-houses.
  This API was then illustrated by example apps, one for Android and one web-based.
  This was in a Windows Embedded environment, with C\# (for back-end) and HTML/CSS/javascript + java for the example apps
}

\cventry{April, 2012 -- July, 2012}{Internship: linux kernel development}{IRIT}
{UPS Toulouse}{France}
{
  Linux kernel development around performance counters and other statistics about the hardware.
  These were inputs to algorithms the research lab simulated to adjust the CPU frequency.
  After development I performed some tests on a French grid computing network, Grid'5000.
}


\section{Education}
\cventry{2013 -- 2014}{Paul Sabatier University}{Toulouse, France}{CAMSI (Informatics, Systems and Machine Architecture Concepts)}{Master degree}
{
  \begin{itemize}
  \item rank 1/18 (Valedictorian)
  \item overall score > 85\%
  \end{itemize}
}

\end{document}
